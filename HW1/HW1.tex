% Options for packages loaded elsewhere
\PassOptionsToPackage{unicode}{hyperref}
\PassOptionsToPackage{hyphens}{url}
%
\documentclass[
]{article}
\usepackage{amsmath,amssymb}
\usepackage{iftex}
\ifPDFTeX
  \usepackage[T1]{fontenc}
  \usepackage[utf8]{inputenc}
  \usepackage{textcomp} % provide euro and other symbols
\else % if luatex or xetex
  \usepackage{unicode-math} % this also loads fontspec
  \defaultfontfeatures{Scale=MatchLowercase}
  \defaultfontfeatures[\rmfamily]{Ligatures=TeX,Scale=1}
\fi
\usepackage{lmodern}
\ifPDFTeX\else
  % xetex/luatex font selection
\fi
% Use upquote if available, for straight quotes in verbatim environments
\IfFileExists{upquote.sty}{\usepackage{upquote}}{}
\IfFileExists{microtype.sty}{% use microtype if available
  \usepackage[]{microtype}
  \UseMicrotypeSet[protrusion]{basicmath} % disable protrusion for tt fonts
}{}
\makeatletter
\@ifundefined{KOMAClassName}{% if non-KOMA class
  \IfFileExists{parskip.sty}{%
    \usepackage{parskip}
  }{% else
    \setlength{\parindent}{0pt}
    \setlength{\parskip}{6pt plus 2pt minus 1pt}}
}{% if KOMA class
  \KOMAoptions{parskip=half}}
\makeatother
\usepackage{xcolor}
\usepackage[margin=1in]{geometry}
\usepackage{color}
\usepackage{fancyvrb}
\newcommand{\VerbBar}{|}
\newcommand{\VERB}{\Verb[commandchars=\\\{\}]}
\DefineVerbatimEnvironment{Highlighting}{Verbatim}{commandchars=\\\{\}}
% Add ',fontsize=\small' for more characters per line
\usepackage{framed}
\definecolor{shadecolor}{RGB}{248,248,248}
\newenvironment{Shaded}{\begin{snugshade}}{\end{snugshade}}
\newcommand{\AlertTok}[1]{\textcolor[rgb]{0.94,0.16,0.16}{#1}}
\newcommand{\AnnotationTok}[1]{\textcolor[rgb]{0.56,0.35,0.01}{\textbf{\textit{#1}}}}
\newcommand{\AttributeTok}[1]{\textcolor[rgb]{0.13,0.29,0.53}{#1}}
\newcommand{\BaseNTok}[1]{\textcolor[rgb]{0.00,0.00,0.81}{#1}}
\newcommand{\BuiltInTok}[1]{#1}
\newcommand{\CharTok}[1]{\textcolor[rgb]{0.31,0.60,0.02}{#1}}
\newcommand{\CommentTok}[1]{\textcolor[rgb]{0.56,0.35,0.01}{\textit{#1}}}
\newcommand{\CommentVarTok}[1]{\textcolor[rgb]{0.56,0.35,0.01}{\textbf{\textit{#1}}}}
\newcommand{\ConstantTok}[1]{\textcolor[rgb]{0.56,0.35,0.01}{#1}}
\newcommand{\ControlFlowTok}[1]{\textcolor[rgb]{0.13,0.29,0.53}{\textbf{#1}}}
\newcommand{\DataTypeTok}[1]{\textcolor[rgb]{0.13,0.29,0.53}{#1}}
\newcommand{\DecValTok}[1]{\textcolor[rgb]{0.00,0.00,0.81}{#1}}
\newcommand{\DocumentationTok}[1]{\textcolor[rgb]{0.56,0.35,0.01}{\textbf{\textit{#1}}}}
\newcommand{\ErrorTok}[1]{\textcolor[rgb]{0.64,0.00,0.00}{\textbf{#1}}}
\newcommand{\ExtensionTok}[1]{#1}
\newcommand{\FloatTok}[1]{\textcolor[rgb]{0.00,0.00,0.81}{#1}}
\newcommand{\FunctionTok}[1]{\textcolor[rgb]{0.13,0.29,0.53}{\textbf{#1}}}
\newcommand{\ImportTok}[1]{#1}
\newcommand{\InformationTok}[1]{\textcolor[rgb]{0.56,0.35,0.01}{\textbf{\textit{#1}}}}
\newcommand{\KeywordTok}[1]{\textcolor[rgb]{0.13,0.29,0.53}{\textbf{#1}}}
\newcommand{\NormalTok}[1]{#1}
\newcommand{\OperatorTok}[1]{\textcolor[rgb]{0.81,0.36,0.00}{\textbf{#1}}}
\newcommand{\OtherTok}[1]{\textcolor[rgb]{0.56,0.35,0.01}{#1}}
\newcommand{\PreprocessorTok}[1]{\textcolor[rgb]{0.56,0.35,0.01}{\textit{#1}}}
\newcommand{\RegionMarkerTok}[1]{#1}
\newcommand{\SpecialCharTok}[1]{\textcolor[rgb]{0.81,0.36,0.00}{\textbf{#1}}}
\newcommand{\SpecialStringTok}[1]{\textcolor[rgb]{0.31,0.60,0.02}{#1}}
\newcommand{\StringTok}[1]{\textcolor[rgb]{0.31,0.60,0.02}{#1}}
\newcommand{\VariableTok}[1]{\textcolor[rgb]{0.00,0.00,0.00}{#1}}
\newcommand{\VerbatimStringTok}[1]{\textcolor[rgb]{0.31,0.60,0.02}{#1}}
\newcommand{\WarningTok}[1]{\textcolor[rgb]{0.56,0.35,0.01}{\textbf{\textit{#1}}}}
\usepackage{graphicx}
\makeatletter
\def\maxwidth{\ifdim\Gin@nat@width>\linewidth\linewidth\else\Gin@nat@width\fi}
\def\maxheight{\ifdim\Gin@nat@height>\textheight\textheight\else\Gin@nat@height\fi}
\makeatother
% Scale images if necessary, so that they will not overflow the page
% margins by default, and it is still possible to overwrite the defaults
% using explicit options in \includegraphics[width, height, ...]{}
\setkeys{Gin}{width=\maxwidth,height=\maxheight,keepaspectratio}
% Set default figure placement to htbp
\makeatletter
\def\fps@figure{htbp}
\makeatother
\setlength{\emergencystretch}{3em} % prevent overfull lines
\providecommand{\tightlist}{%
  \setlength{\itemsep}{0pt}\setlength{\parskip}{0pt}}
\setcounter{secnumdepth}{-\maxdimen} % remove section numbering
\ifLuaTeX
  \usepackage{selnolig}  % disable illegal ligatures
\fi
\usepackage{bookmark}
\IfFileExists{xurl.sty}{\usepackage{xurl}}{} % add URL line breaks if available
\urlstyle{same}
\hypersetup{
  pdftitle={111208094\_hw1},
  hidelinks,
  pdfcreator={LaTeX via pandoc}}

\title{111208094\_hw1}
\author{}
\date{\vspace{-2.5em}}

\begin{document}
\maketitle

\begin{Shaded}
\begin{Highlighting}[]
\FunctionTok{options}\NormalTok{(}\AttributeTok{repos =} \FunctionTok{c}\NormalTok{(}\AttributeTok{CRAN =} \StringTok{"https://cran.rstudio.com/"}\NormalTok{))}
\FunctionTok{install.packages}\NormalTok{(}\StringTok{"tinytex"}\NormalTok{)}
\end{Highlighting}
\end{Shaded}

\begin{verbatim}
## 將程式套件安載入 'C:/Users/Ava/AppData/Local/R/win-library/4.4'
## (因為 'lib' 沒有被指定)
\end{verbatim}

\begin{verbatim}
## 程式套件 'tinytex' 開啟成功,MD5 和檢查也透過
## 
## 下載的二進位程式套件在
##  C:\Users\Ava\AppData\Local\Temp\Rtmp8qhq90\downloaded_packages 裡
\end{verbatim}

\begin{Shaded}
\begin{Highlighting}[]
\NormalTok{tinytex}\SpecialCharTok{::}\FunctionTok{install\_tinytex}\NormalTok{(}\AttributeTok{force =} \ConstantTok{TRUE}\NormalTok{)}
\end{Highlighting}
\end{Shaded}

\begin{verbatim}
## tlmgr install tlgpg
\end{verbatim}

\begin{verbatim}
## tlmgr update --self
\end{verbatim}

\begin{verbatim}
## tlmgr install tlgpg
\end{verbatim}

\begin{verbatim}
## tlmgr --repository http://www.preining.info/tlgpg/ install tlgpg
\end{verbatim}

\begin{verbatim}
## tlmgr option repository "https://tw.mirrors.cicku.me/ctan/systems/texlive/tlnet"
\end{verbatim}

\begin{verbatim}
## tlmgr update --list
\end{verbatim}

\begin{enumerate}
\def\labelenumi{\arabic{enumi}.}
\tightlist
\item
  請讀入資料:stock = read\_csv(``stock.csv'')
  資料為2023/12/1\textasciitilde2024/3/7,5個股票的股價,變數為:id\_name:每檔股票的證券代碼與名稱,如:2330台積電,表示證券代碼2330,公司名稱為台積電。type:open
  代表開盤價,close 代表收盤價。
  2023/12/1:為該天交易價格(剩下日期變數依此類推)。
  請用tidyr提到的gather,spread,sperate等函數指令,將資料整理成下方「tidy」格式
\end{enumerate}

\begin{Shaded}
\begin{Highlighting}[]
\CommentTok{\#By myself (我寫好爛好長)}
\FunctionTok{install.packages}\NormalTok{(}\StringTok{"readr"}\NormalTok{)}
\end{Highlighting}
\end{Shaded}

\begin{verbatim}
## 將程式套件安載入 'C:/Users/Ava/AppData/Local/R/win-library/4.4'
## (因為 'lib' 沒有被指定)
\end{verbatim}

\begin{verbatim}
## 程式套件 'readr' 開啟成功,MD5 和檢查也透過
## 
## 下載的二進位程式套件在
##  C:\Users\Ava\AppData\Local\Temp\Rtmp8qhq90\downloaded_packages 裡
\end{verbatim}

\begin{Shaded}
\begin{Highlighting}[]
\FunctionTok{library}\NormalTok{(readr)}
\FunctionTok{library}\NormalTok{(tidyr)}
\NormalTok{stock\_short }\OtherTok{\textless{}{-}} \FunctionTok{read\_csv}\NormalTok{(}\StringTok{"C:}\SpecialCharTok{\textbackslash{}\textbackslash{}}\StringTok{Users}\SpecialCharTok{\textbackslash{}\textbackslash{}}\StringTok{Ava}\SpecialCharTok{\textbackslash{}\textbackslash{}}\StringTok{Desktop}\SpecialCharTok{\textbackslash{}\textbackslash{}}\StringTok{R}\SpecialCharTok{\textbackslash{}\textbackslash{}}\StringTok{HW1}\SpecialCharTok{\textbackslash{}\textbackslash{}}\StringTok{stock.csv"}\NormalTok{)}
\end{Highlighting}
\end{Shaded}

\begin{verbatim}
## Rows: 10 Columns: 63
\end{verbatim}

\begin{verbatim}
## -- Column specification --------------------------------------------------------
## Delimiter: ","
## chr  (2): id_name, type
## dbl (61): 2024/3/7, 2024/3/6, 2024/3/5, 2024/3/4, 2024/3/1, 2024/2/29, 2024/...
## 
## i Use `spec()` to retrieve the full column specification for this data.
## i Specify the column types or set `show_col_types = FALSE` to quiet this message.
\end{verbatim}

\begin{Shaded}
\begin{Highlighting}[]
\NormalTok{stock\_short1 }\OtherTok{\textless{}{-}}\FunctionTok{separate}\NormalTok{(stock\_short,id\_name,}\AttributeTok{into =} \FunctionTok{c}\NormalTok{(}\StringTok{"id"}\NormalTok{,}\StringTok{"name"}\NormalTok{),}\AttributeTok{sep =} \StringTok{" "}\NormalTok{)}
\NormalTok{stock\_short2 }\OtherTok{\textless{}{-}} \FunctionTok{gather}\NormalTok{(stock\_short1,}\AttributeTok{key =} \StringTok{"time"}\NormalTok{ ,}\AttributeTok{value =} \StringTok{"value"}\NormalTok{, }\SpecialCharTok{{-}}\NormalTok{id,}\SpecialCharTok{{-}}\NormalTok{name,}\SpecialCharTok{{-}}\NormalTok{type)}
\NormalTok{stock\_short3 }\OtherTok{\textless{}{-}}\FunctionTok{separate}\NormalTok{(stock\_short2,time,}\AttributeTok{into =} \FunctionTok{c}\NormalTok{(}\StringTok{"year"}\NormalTok{,}\StringTok{"month"}\NormalTok{,}\StringTok{"date"}\NormalTok{),}\AttributeTok{sep =} \StringTok{"/"}\NormalTok{)}
\NormalTok{stock\_long }\OtherTok{\textless{}{-}} \FunctionTok{pivot\_wider}\NormalTok{(stock\_short3, }\AttributeTok{names\_from =}\NormalTok{ type, }\AttributeTok{values\_from =}\NormalTok{ value) }
\FunctionTok{print}\NormalTok{(stock\_long) }\CommentTok{\#answer}
\end{Highlighting}
\end{Shaded}

\begin{verbatim}
## # A tibble: 305 x 7
##    id    name   year  month date    open  close
##    <chr> <chr>  <chr> <chr> <chr>  <dbl>  <dbl>
##  1 2330  台積電 2024  3     7      755    760  
##  2 2002  中鋼   2024  3     7       24     24.0
##  3 2885  元大金 2024  3     7       27.3   28.1
##  4 9904  寶成   2024  3     7       30.3   30.4
##  5 2454  聯發科 2024  3     7     1220   1230  
##  6 2330  台積電 2024  3     6      718    735  
##  7 2002  中鋼   2024  3     6       24.0   24.0
##  8 2885  元大金 2024  3     6       27.4   27.5
##  9 9904  寶成   2024  3     6       30.5   30.3
## 10 2454  聯發科 2024  3     6     1145   1190  
## # i 295 more rows
\end{verbatim}

\begin{enumerate}
\def\labelenumi{\arabic{enumi}.}
\setcounter{enumi}{1}
\tightlist
\item
  電商公司,有三個資料集合:
\end{enumerate}

sales.df:產品銷售狀況 (``salesID''銷售紀錄編號,``Store''店家編號,
``Product''產品編號,``Client''顧客編號,``UnitPrice''單價,``Quantity''購買數量,``Region''顧客國家
)

client.df:顧客的個人資料
(``Client''顧客編號,``Age''年紀,``Membership''會員等級,
``Gender''性別 )

prod.df:產品的相關資料(``Item'' 代號_產品)
請用tidyverse套件裡學到的方法,分析

(1.) prod.df
裡將兩個變數,誤紀錄為在同一個column,其將其分開為兩個變數Product(數字部分)及Item(商品部分),取代原prod.df。
(2.) 將3個報表合併為full.table (3.) 在full.table.
新增一個變數「總消費」為spend = UnitPrice*Quantity (4.)
在full.table將會員等級分組,其中gold和diamond的顧客為一組,其他等級的為一組,針對兩組客戶進行比較介紹(例如平均年紀、性別、國家、消費情況差異等)。
(5.)
在full.table針對女性客戶進行分析(例如平均年紀、國家、消費情況等),並對他們在不同產品的「總消費」畫圖分析。

\begin{Shaded}
\begin{Highlighting}[]
\CommentTok{\#資料匯入}
\NormalTok{client\_list }\OtherTok{\textless{}{-}} \FunctionTok{read\_csv}\NormalTok{(}\StringTok{"C:}\SpecialCharTok{\textbackslash{}\textbackslash{}}\StringTok{Users}\SpecialCharTok{\textbackslash{}\textbackslash{}}\StringTok{Ava}\SpecialCharTok{\textbackslash{}\textbackslash{}}\StringTok{Desktop}\SpecialCharTok{\textbackslash{}\textbackslash{}}\StringTok{R}\SpecialCharTok{\textbackslash{}\textbackslash{}}\StringTok{HW1}\SpecialCharTok{\textbackslash{}\textbackslash{}}\StringTok{client\_list.csv"}\NormalTok{)}
\end{Highlighting}
\end{Shaded}

\begin{verbatim}
## New names:
## Rows: 20 Columns: 5
## -- Column specification
## -------------------------------------------------------- Delimiter: "," chr
## (2): Membership, Gender dbl (3): ...1, Client, Age
## i Use `spec()` to retrieve the full column specification for this data. i
## Specify the column types or set `show_col_types = FALSE` to quiet this message.
## * `` -> `...1`
\end{verbatim}

\begin{Shaded}
\begin{Highlighting}[]
\NormalTok{salesdata }\OtherTok{\textless{}{-}} \FunctionTok{read\_csv}\NormalTok{(}\StringTok{"C:}\SpecialCharTok{\textbackslash{}\textbackslash{}}\StringTok{Users}\SpecialCharTok{\textbackslash{}\textbackslash{}}\StringTok{Ava}\SpecialCharTok{\textbackslash{}\textbackslash{}}\StringTok{Desktop}\SpecialCharTok{\textbackslash{}\textbackslash{}}\StringTok{R}\SpecialCharTok{\textbackslash{}\textbackslash{}}\StringTok{HW1}\SpecialCharTok{\textbackslash{}\textbackslash{}}\StringTok{salesdata.csv"}\NormalTok{)}
\end{Highlighting}
\end{Shaded}

\begin{verbatim}
## New names:
## Rows: 100 Columns: 7
## -- Column specification
## -------------------------------------------------------- Delimiter: "," chr
## (2): Store, Region dbl (5): ...1, Product, Client, UnitPrice, Quantity
## i Use `spec()` to retrieve the full column specification for this data. i
## Specify the column types or set `show_col_types = FALSE` to quiet this message.
## * `` -> `...1`
\end{verbatim}

\begin{Shaded}
\begin{Highlighting}[]
\NormalTok{product\_list }\OtherTok{\textless{}{-}} \FunctionTok{read\_csv}\NormalTok{(}\StringTok{"C:}\SpecialCharTok{\textbackslash{}\textbackslash{}}\StringTok{Users}\SpecialCharTok{\textbackslash{}\textbackslash{}}\StringTok{Ava}\SpecialCharTok{\textbackslash{}\textbackslash{}}\StringTok{Desktop}\SpecialCharTok{\textbackslash{}\textbackslash{}}\StringTok{R}\SpecialCharTok{\textbackslash{}\textbackslash{}}\StringTok{HW1}\SpecialCharTok{\textbackslash{}\textbackslash{}}\StringTok{product\_list.csv"}\NormalTok{)}
\end{Highlighting}
\end{Shaded}

\begin{verbatim}
## Rows: 6 Columns: 1
## -- Column specification --------------------------------------------------------
## Delimiter: ","
## chr (1): Item
## 
## i Use `spec()` to retrieve the full column specification for this data.
## i Specify the column types or set `show_col_types = FALSE` to quiet this message.
\end{verbatim}

\begin{Shaded}
\begin{Highlighting}[]
\CommentTok{\#(1.)}
\NormalTok{product\_list1 }\OtherTok{\textless{}{-}} \FunctionTok{separate}\NormalTok{(product\_list,Item,}\AttributeTok{into =} \FunctionTok{c}\NormalTok{(}\StringTok{"Product"}\NormalTok{,}\StringTok{"Item"}\NormalTok{),}\AttributeTok{sep =} \StringTok{"\_"}\NormalTok{,,}\AttributeTok{convert=}\ConstantTok{TRUE}\NormalTok{)}
\NormalTok{product\_list1}
\end{Highlighting}
\end{Shaded}

\begin{verbatim}
## # A tibble: 6 x 2
##   Product Item      
##     <int> <chr>     
## 1     101 iPhone    
## 2     102 iPad      
## 3     103 MacBook   
## 4     104 iMac      
## 5     105 AirPods   
## 6     106 AppleWatch
\end{verbatim}

\begin{verbatim}
## 
## 載入套件:'dplyr'
\end{verbatim}

\begin{verbatim}
## 下列物件被遮斷自 'package:stats':
## 
##     filter, lag
\end{verbatim}

\begin{verbatim}
## 下列物件被遮斷自 'package:base':
## 
##     intersect, setdiff, setequal, union
\end{verbatim}

\begin{verbatim}
## # A tibble: 100 x 12
##    ...1.x Store Product Client UnitPrice Quantity Region ...1.y   Age Membership
##     <dbl> <chr>   <dbl>  <dbl>     <dbl>    <dbl> <chr>   <dbl> <dbl> <chr>     
##  1      1 A         103      1        10       72 Brazil      1    36 basic     
##  2     25 A         103      1        18       26 Korea       1    36 basic     
##  3     28 C         105      1        11        4 Japan       1    36 basic     
##  4     10 A         103      2        14       58 China       2    46 diamond   
##  5     11 C         101      2         8       35 Brazil      2    46 diamond   
##  6     41 B         105      2         4       76 Korea       2    46 diamond   
##  7     59 B         101      2         7       58 Korea       2    46 diamond   
##  8     67 B         105      2        14       98 Thail~      2    46 diamond   
##  9     84 C         105      2        14       59 Germa~      2    46 diamond   
## 10     97 B         105      2         9       20 Brazil      2    46 diamond   
## # i 90 more rows
## # i 2 more variables: Gender <chr>, Item <chr>
\end{verbatim}

\begin{Shaded}
\begin{Highlighting}[]
\CommentTok{\#(3.)}
\FunctionTok{library}\NormalTok{(tidyverse)}
\end{Highlighting}
\end{Shaded}

\begin{verbatim}
## -- Attaching core tidyverse packages ------------------------ tidyverse 2.0.0 --
## v forcats   1.0.0     v purrr     1.0.2
## v ggplot2   3.5.1     v stringr   1.5.1
## v lubridate 1.9.3     v tibble    3.2.1
## -- Conflicts ------------------------------------------ tidyverse_conflicts() --
## x dplyr::filter() masks stats::filter()
## x dplyr::lag()    masks stats::lag()
## i Use the conflicted package (<http://conflicted.r-lib.org/>) to force all conflicts to become errors
\end{verbatim}

\begin{Shaded}
\begin{Highlighting}[]
\NormalTok{full\_table}\SpecialCharTok{\%\textgreater{}\%}
  \FunctionTok{mutate}\NormalTok{( }\AttributeTok{Spend =}\NormalTok{ UnitPrice}\SpecialCharTok{*}\NormalTok{Quantity ) }\OtherTok{{-}\textgreater{}}\NormalTok{ full\_table1}
\FunctionTok{library}\NormalTok{(writexl)}
\FunctionTok{write\_xlsx}\NormalTok{(full\_table1, }\AttributeTok{path =} \StringTok{"C:/Users/Ava/Downloads/full\_table1.xlsx"}\NormalTok{) }\CommentTok{\#保存下來方便之後跑EDA!!}
\NormalTok{full\_table1}
\end{Highlighting}
\end{Shaded}

\begin{verbatim}
## # A tibble: 100 x 13
##    ...1.x Store Product Client UnitPrice Quantity Region ...1.y   Age Membership
##     <dbl> <chr>   <dbl>  <dbl>     <dbl>    <dbl> <chr>   <dbl> <dbl> <chr>     
##  1      1 A         103      1        10       72 Brazil      1    36 basic     
##  2     25 A         103      1        18       26 Korea       1    36 basic     
##  3     28 C         105      1        11        4 Japan       1    36 basic     
##  4     10 A         103      2        14       58 China       2    46 diamond   
##  5     11 C         101      2         8       35 Brazil      2    46 diamond   
##  6     41 B         105      2         4       76 Korea       2    46 diamond   
##  7     59 B         101      2         7       58 Korea       2    46 diamond   
##  8     67 B         105      2        14       98 Thail~      2    46 diamond   
##  9     84 C         105      2        14       59 Germa~      2    46 diamond   
## 10     97 B         105      2         9       20 Brazil      2    46 diamond   
## # i 90 more rows
## # i 3 more variables: Gender <chr>, Item <chr>, Spend <dbl>
\end{verbatim}

\begin{Shaded}
\begin{Highlighting}[]
\CommentTok{\#(4.) EDA Start!!!}
\CommentTok{\#分組}
\NormalTok{full\_table2 }\OtherTok{\textless{}{-}}\NormalTok{ full\_table1 }\SpecialCharTok{\%\textgreater{}\%}
  \FunctionTok{mutate}\NormalTok{(}\AttributeTok{Group =} \FunctionTok{ifelse}\NormalTok{(Membership }\SpecialCharTok{\%in\%} \FunctionTok{c}\NormalTok{(}\StringTok{"gold"}\NormalTok{, }\StringTok{"diamond"}\NormalTok{), }\StringTok{"Gold \& Diamond"}\NormalTok{, }\StringTok{"Other"}\NormalTok{))}
\NormalTok{full\_table2}
\end{Highlighting}
\end{Shaded}

\begin{verbatim}
## # A tibble: 100 x 14
##    ...1.x Store Product Client UnitPrice Quantity Region ...1.y   Age Membership
##     <dbl> <chr>   <dbl>  <dbl>     <dbl>    <dbl> <chr>   <dbl> <dbl> <chr>     
##  1      1 A         103      1        10       72 Brazil      1    36 basic     
##  2     25 A         103      1        18       26 Korea       1    36 basic     
##  3     28 C         105      1        11        4 Japan       1    36 basic     
##  4     10 A         103      2        14       58 China       2    46 diamond   
##  5     11 C         101      2         8       35 Brazil      2    46 diamond   
##  6     41 B         105      2         4       76 Korea       2    46 diamond   
##  7     59 B         101      2         7       58 Korea       2    46 diamond   
##  8     67 B         105      2        14       98 Thail~      2    46 diamond   
##  9     84 C         105      2        14       59 Germa~      2    46 diamond   
## 10     97 B         105      2         9       20 Brazil      2    46 diamond   
## # i 90 more rows
## # i 4 more variables: Gender <chr>, Item <chr>, Spend <dbl>, Group <chr>
\end{verbatim}

\begin{Shaded}
\begin{Highlighting}[]
\CommentTok{\#年紀}
\NormalTok{age\_comparison }\OtherTok{\textless{}{-}}\NormalTok{ full\_table2 }\SpecialCharTok{\%\textgreater{}\%}
  \FunctionTok{group\_by}\NormalTok{(Group) }\SpecialCharTok{\%\textgreater{}\%}
  \FunctionTok{summarise}\NormalTok{(}\AttributeTok{Average\_Age =} \FunctionTok{mean}\NormalTok{(Age, }\AttributeTok{na.rm =} \ConstantTok{TRUE}\NormalTok{))}
\NormalTok{age\_comparison}
\end{Highlighting}
\end{Shaded}

\begin{verbatim}
## # A tibble: 2 x 2
##   Group          Average_Age
##   <chr>                <dbl>
## 1 Gold & Diamond        41.8
## 2 Other                 40.9
\end{verbatim}

\begin{Shaded}
\begin{Highlighting}[]
\CommentTok{\#性別}
\NormalTok{gender\_comparison }\OtherTok{\textless{}{-}}\NormalTok{ full\_table2 }\SpecialCharTok{\%\textgreater{}\%}
  \FunctionTok{group\_by}\NormalTok{(Group, Gender) }\SpecialCharTok{\%\textgreater{}\%}
  \FunctionTok{summarise}\NormalTok{(}\AttributeTok{Count =} \FunctionTok{n}\NormalTok{()) }\SpecialCharTok{\%\textgreater{}\%}
  \FunctionTok{mutate}\NormalTok{(}\AttributeTok{Percentage =}\NormalTok{ Count }\SpecialCharTok{/} \FunctionTok{sum}\NormalTok{(Count) }\SpecialCharTok{*} \DecValTok{100}\NormalTok{)}
\end{Highlighting}
\end{Shaded}

\begin{verbatim}
## `summarise()` has grouped output by 'Group'. You can override using the
## `.groups` argument.
\end{verbatim}

\begin{Shaded}
\begin{Highlighting}[]
\NormalTok{gender\_comparison}
\end{Highlighting}
\end{Shaded}

\begin{verbatim}
## # A tibble: 4 x 4
## # Groups:   Group [2]
##   Group          Gender Count Percentage
##   <chr>          <chr>  <int>      <dbl>
## 1 Gold & Diamond female    43       75.4
## 2 Gold & Diamond male      14       24.6
## 3 Other          female    21       48.8
## 4 Other          male      22       51.2
\end{verbatim}

\begin{Shaded}
\begin{Highlighting}[]
\CommentTok{\#國家}
\NormalTok{country\_comparison }\OtherTok{\textless{}{-}}\NormalTok{ full\_table2 }\SpecialCharTok{\%\textgreater{}\%}
  \FunctionTok{group\_by}\NormalTok{(Group, Region) }\SpecialCharTok{\%\textgreater{}\%}
  \FunctionTok{summarise}\NormalTok{(}\AttributeTok{Count =} \FunctionTok{n}\NormalTok{()) }\SpecialCharTok{\%\textgreater{}\%}
  \FunctionTok{arrange}\NormalTok{(Group, }\FunctionTok{desc}\NormalTok{(Count))}
\end{Highlighting}
\end{Shaded}

\begin{verbatim}
## `summarise()` has grouped output by 'Group'. You can override using the
## `.groups` argument.
\end{verbatim}

\begin{Shaded}
\begin{Highlighting}[]
\NormalTok{country\_comparison}
\end{Highlighting}
\end{Shaded}

\begin{verbatim}
## # A tibble: 20 x 3
## # Groups:   Group [2]
##    Group          Region   Count
##    <chr>          <chr>    <int>
##  1 Gold & Diamond China        8
##  2 Gold & Diamond Korea        8
##  3 Gold & Diamond Germany      6
##  4 Gold & Diamond Taiwan       6
##  5 Gold & Diamond Thailand     6
##  6 Gold & Diamond Brazil       5
##  7 Gold & Diamond France       5
##  8 Gold & Diamond Spain        5
##  9 Gold & Diamond USA          5
## 10 Gold & Diamond Japan        3
## 11 Other          Korea       11
## 12 Other          Taiwan       6
## 13 Other          Spain        5
## 14 Other          USA          5
## 15 Other          Brazil       4
## 16 Other          France       4
## 17 Other          China        3
## 18 Other          Germany      2
## 19 Other          Thailand     2
## 20 Other          Japan        1
\end{verbatim}

\begin{Shaded}
\begin{Highlighting}[]
\CommentTok{\#消費差異}
\NormalTok{spend\_comparison }\OtherTok{\textless{}{-}}\NormalTok{ full\_table2 }\SpecialCharTok{\%\textgreater{}\%}
  \FunctionTok{group\_by}\NormalTok{(Group) }\SpecialCharTok{\%\textgreater{}\%}
  \FunctionTok{summarise}\NormalTok{(}\AttributeTok{Average\_Spend =} \FunctionTok{mean}\NormalTok{(Spend, }\AttributeTok{na.rm =} \ConstantTok{TRUE}\NormalTok{),}
            \AttributeTok{Total\_Spend =} \FunctionTok{sum}\NormalTok{(Spend, }\AttributeTok{na.rm =} \ConstantTok{TRUE}\NormalTok{),}
            \AttributeTok{Std\_Dev\_Spend =} \FunctionTok{sd}\NormalTok{(Spend, }\AttributeTok{na.rm =} \ConstantTok{TRUE}\NormalTok{))}
\end{Highlighting}
\end{Shaded}

\begin{Shaded}
\begin{Highlighting}[]
\CommentTok{\#(5.)}
\NormalTok{full\_table2\_female }\OtherTok{\textless{}{-}}\NormalTok{ full\_table2[full\_table2}\SpecialCharTok{$}\NormalTok{Gender }\SpecialCharTok{==}\StringTok{"female"}\NormalTok{,]}
\FunctionTok{write\_xlsx}\NormalTok{(full\_table2\_female, }\AttributeTok{path =} \StringTok{"C:/Users/Ava/Downloads/full\_table2\_female.xlsx"}\NormalTok{)}

\CommentTok{\#年紀}
\NormalTok{age\_comparison\_female }\OtherTok{\textless{}{-}}\NormalTok{ full\_table2\_female }\SpecialCharTok{\%\textgreater{}\%}
  \FunctionTok{summarise}\NormalTok{(}\AttributeTok{Average\_Age =} \FunctionTok{mean}\NormalTok{(Age, }\AttributeTok{na.rm =} \ConstantTok{TRUE}\NormalTok{))}

\FunctionTok{pie}\NormalTok{(}\FunctionTok{table}\NormalTok{(full\_table2\_female}\SpecialCharTok{$}\NormalTok{Age))}
\end{Highlighting}
\end{Shaded}

\includegraphics{HW1_files/figure-latex/unnamed-chunk-5-1.pdf}

\begin{Shaded}
\begin{Highlighting}[]
\CommentTok{\#國家}
\NormalTok{country\_comparison\_female }\OtherTok{\textless{}{-}}\NormalTok{ full\_table2\_female }\SpecialCharTok{\%\textgreater{}\%}
  \FunctionTok{group\_by}\NormalTok{(Region) }\SpecialCharTok{\%\textgreater{}\%}
  \FunctionTok{summarise}\NormalTok{(}\AttributeTok{Count =} \FunctionTok{n}\NormalTok{()) }\SpecialCharTok{\%\textgreater{}\%}
  \FunctionTok{arrange}\NormalTok{(}\FunctionTok{desc}\NormalTok{(Count))}

\FunctionTok{pie}\NormalTok{(}\FunctionTok{table}\NormalTok{(full\_table2\_female}\SpecialCharTok{$}\NormalTok{Region)) }
\end{Highlighting}
\end{Shaded}

\includegraphics{HW1_files/figure-latex/unnamed-chunk-5-2.pdf}

\begin{Shaded}
\begin{Highlighting}[]
\CommentTok{\#消費差異}
\NormalTok{spend\_comparison\_female }\OtherTok{\textless{}{-}}\NormalTok{ full\_table2\_female }\SpecialCharTok{\%\textgreater{}\%}
  \FunctionTok{summarise}\NormalTok{(}\AttributeTok{Average\_Spend =} \FunctionTok{mean}\NormalTok{(Spend, }\AttributeTok{na.rm =} \ConstantTok{TRUE}\NormalTok{),}
            \AttributeTok{Total\_Spend =} \FunctionTok{sum}\NormalTok{(Spend, }\AttributeTok{na.rm =} \ConstantTok{TRUE}\NormalTok{),}
            \AttributeTok{Std\_Dev\_Spend =} \FunctionTok{sd}\NormalTok{(Spend, }\AttributeTok{na.rm =} \ConstantTok{TRUE}\NormalTok{))}

\CommentTok{\#不同產品的「總消費」畫圖分析}
\FunctionTok{library}\NormalTok{(ggplot2)}
\NormalTok{p }\OtherTok{\textless{}{-}} \FunctionTok{ggplot}\NormalTok{(full\_table2\_female, }\FunctionTok{aes}\NormalTok{(}\AttributeTok{x =}\NormalTok{ Spend , }\AttributeTok{fill =}\NormalTok{ Item)) }\SpecialCharTok{+} \FunctionTok{geom\_density}\NormalTok{(}\AttributeTok{alpha =} \FloatTok{0.7}\NormalTok{)}
\FunctionTok{plot}\NormalTok{(p, }\AttributeTok{labels =} \StringTok{"AUTO"}\NormalTok{)}
\end{Highlighting}
\end{Shaded}

\includegraphics{HW1_files/figure-latex/unnamed-chunk-5-3.pdf}

\begin{Shaded}
\begin{Highlighting}[]
\NormalTok{plot}
\end{Highlighting}
\end{Shaded}

\begin{verbatim}
## function (x, y, ...) 
## UseMethod("plot")
## <bytecode: 0x0000025e693bbe78>
## <environment: namespace:base>
\end{verbatim}

\end{document}
